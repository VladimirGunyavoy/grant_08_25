\documentclass[12pt,a4paper]{article}
\usepackage[utf8]{inputenc}
\usepackage[russian]{babel}
\usepackage{geometry}
\usepackage{graphicx}
\usepackage{amsmath}
\usepackage{amsfonts}
\usepackage{amssymb}
\usepackage{booktabs}
\usepackage{array}
\usepackage{multirow}
\usepackage{xcolor}
\usepackage{hyperref}
\usepackage{fancyhdr}
\usepackage{titlesec}
\usepackage{enumitem}
\usepackage{setspace}

% Настройки страницы
\geometry{left=2cm,right=2cm,top=2cm,bottom=2cm}

% Настройки гиперссылок
\hypersetup{
    colorlinks=true,
    linkcolor=blue,
    filecolor=magenta,      
    urlcolor=cyan,
    citecolor=green
}

% Настройки заголовков
\titleformat{\section}{\Large\bfseries}{\thesection}{1em}{}
\titleformat{\subsection}{\large\bfseries}{\thesubsection}{1em}{}
\titleformat{\subsubsection}{\normalsize\bfseries}{\thesubsubsection}{1em}{}

% Настройки списков
\setlist[itemize]{leftmargin=*,itemsep=1pt,parsep=0pt}
\setlist[enumerate]{leftmargin=*,itemsep=1pt,parsep=0pt}

% Настройки межстрочного интервала
\onehalfspacing

% Настройки колонтитулов
\pagestyle{fancy}
\fancyhf{}
\fancyhead[L]{\small Описание результатов исследований AIDA-T}
\fancyhead[R]{\small \thepage}
\renewcommand{\headrulewidth}{0.4pt}

\begin{document}

% Титульная страница
\begin{titlepage}
    \centering
    \vspace*{1cm}
    
    {\Huge\bfseries Описание предполагаемых результатов исследований, их научной и практической ценности, а также проекта образовательных программ\par}
    
    \vspace{1.5cm}
    
    {\Large\bfseries Проект: AIDA-T\par}
    {\large (Agrobotic Intelligent Data Analyzer for Tomatoes)\par}
    
    \vspace{1cm}
    
    {\large\bfseries АИДА-Т\par}
    {\large (Агроробототехнический Интеллектуальный Анализатор Данных для Томатов)\par}
    
    \vspace{1.5cm}
    
    \begin{tabular}{ll}
        \textbf{Руководитель:} & Осиненко Павел Валерьевич \\
        \textbf{Сроки реализации:} & 2024-2026 гг. (24 месяца) \\
        \textbf{Бюджет проекта:} & ~50 млн рублей
    \end{tabular}
    
    \vfill
    
    {\large \today\par}
\end{titlepage}

\section{Краткое описание проекта}

Проект АИДА-Т направлен на создание передовой автономной робототехнической системы для интеллектуального мониторинга и диагностики томатов в промышленных теплицах. Основная цель — разработка функционирующего прототипа мобильного робота с системой компьютерного зрения и ИИ, способного автономно перемещаться по теплице, выявлять заболевания растений на ранних стадиях и оценивать урожайность.

\textbf{Актуальность:} Рынок роботизации в АПК России растет на 25-30\% в год, достигнув 15 млрд рублей в 2024 году. Автоматизация позволяет снизить затраты на агрономический контроль до 70\%.

\textbf{Ключевые особенности:}
\begin{itemize}
    \item Уникальная гибридная ходовая система (меканум-колеса + рельсовые колеса)
    \item Телескопическая камерная мачта с активной стабилизацией (0.1-3.0 м)
    \item CNN-алгоритмы диагностики заболеваний с точностью 86-87\%
    \item Гибридный метод оценки урожайности (RANSAC + PointNet) с погрешностью не более 15\%
    \item Полная автономность работы до 12 часов
    \item Стоимость системы: 15-20 млн рублей (период окупаемости 2.5-3 года)
\end{itemize}

\section{Научно-техническая новизна}

\textbf{Техническая новизна аппаратной части:}
\begin{itemize}
    \item \textbf{Гибридная ходовая система} — впервые предложена комбинация меканум-колес и нейлоновых рельсовых колес для автономного движения по различным поверхностям теплицы
    \item \textbf{Активная стабилизация} — точность позиционирования ±1.5 мм на пиксель даже при движении по рельсам
    \item \textbf{Адаптивная камерная мачта} с регулировкой высоты и системой слияния кадров
\end{itemize}

\textbf{Программная новизна:}
\begin{itemize}
    \item \textbf{Гибридный подход к оценке объема плодов} — параллельная обработка данных методами RANSAC и PointNet
    \item \textbf{Специализированная CNN-архитектура} для диагностики заболеваний в условиях переменного освещения
    \item \textbf{Алгоритм автономной навигации} с динамическим переключением между режимами движения
\end{itemize}

\section{Предполагаемые результаты исследований}

\subsection{Аппаратная платформа}
\begin{itemize}
    \item Гибридная мобильная платформа с движением по бетонным покрытиям и рельсовым путям
    \item Система технического зрения с телескопической мачтой и активной стабилизацией
    \item Интегрированный вычислительный блок промышленного класса
    \item Система энергоснабжения на базе Li-ion аккумуляторов (12+ часов)
\end{itemize}

\subsection{Программный комплекс}
\begin{itemize}
    \item CNN для диагностики заболеваний томатов с точностью 86-87\%
    \item Гибридный метод оценки урожайности с погрешностью не более 15\%
    \item Система автономной навигации на базе ROS2
    \item Веб-интерфейс для удаленного мониторинга и управления
\end{itemize}

\subsection{Документация и ИС}
\begin{itemize}
    \item Конструкторская и программная документация
    \item Программы и методики испытаний
    \item 2 заявки на полезные модели
    \item 1 регистрация программы для ЭВМ
\end{itemize}

\section{Научная и практическая ценность}

\subsection{Научная ценность}
Научная новизна заключается в решении междисциплинарных задач на стыке робототехники, компьютерного зрения и агротехнологий:
\begin{itemize}
    \item Концепция гибридной ходовой системы для адаптации к различным поверхностям
    \item Специализированная CNN-архитектура для диагностики в сложных условиях освещения
    \item Алгоритм автономной навигации с динамическим переключением моделей управления
\end{itemize}

Результаты могут быть опубликованы в журналах Q1/Q2 и представлены на конференциях IEEE IROS, ICRA, AAAI.

\subsection{Практическая ценность}

\textbf{Экономический эффект:}
\begin{itemize}
    \item Снижение трудозатрат до 70\%
    \item Повышение урожайности на 15-20\% за счет раннего обнаружения заболеваний
    \item Снижение расхода средств защиты растений на 30-40\%
    \item Период окупаемости 2.5-3 года
\end{itemize}

\textbf{Дополнительные преимущества:}
\begin{itemize}
    \item Полностью российская разработка, превосходящая зарубежные аналоги
    \item Решение проблемы дефицита квалифицированных агрономов
    \item Масштабируемость для других культур и операций
\end{itemize}

\subsection{Команда проекта}
\textbf{Руководитель:} Осиненко Павел Валерьевич, к.т.н., Сколтех

\textbf{Состав:} 4 основных специалиста (руководитель проекта, CV-разработчик, научный консультант, математический инженер) + привлекаемые эксперты

\textbf{Партнеры:} Сколтех, МФТИ, ведущие тепличные комплексы России

\section{Анализ рынка и конкурентные преимущества}

\subsection{Объем рынка}
\begin{itemize}
    \item Российский рынок защищенного грунта: 280 млрд рублей (2024), рост 12-15\% в год
    \item Рынок роботизации в АПК России: 15 млрд рублей (2024), рост 25-30\% в год
    \item Доступный рынок для AIDA-T: 2.1 млрд рублей (2025) → 4.8 млрд рублей (2030)
    \item Целевая доля рынка к 2030 году: 3-5\% (150-250 млн рублей)
\end{itemize}

\subsection{Конкурентные преимущества}
\begin{table}[h]
\centering
\begin{tabular}{|l|c|c|c|}
\hline
\textbf{Параметр} & \textbf{AIDA-T} & \textbf{Bosch DeepField} & \textbf{Iron Ox} \\
\hline
Точность диагностики & 86-87\% & 75-80\% & 70-75\% \\
\hline
Ошибка оценки объема & $\leq$15\% & 15-20\% & 12-15\% \\
\hline
Точность позиционирования & ±1.5 мм & ±5 мм & ±3 мм \\
\hline
Тип ходовой системы & Гибридная & Колесная & Рельсовая \\
\hline
Стоимость & 15-20 млн руб & \$150-200 тыс & \$120-180 тыс \\
\hline
\end{tabular}
\caption{Сравнение с основными конкурентами}
\end{table}

\section{Проект образовательных программ}

\subsection{Магистерская программа}
\textbf{Название:} ``Интеллектуальная робототехника в агропромышленном комплексе''

\textbf{Ключевые курсы:}
\begin{itemize}
    \item Мобильная робототехника и навигационные системы в сельском хозяйстве
    \item Компьютерное зрение и машинное обучение для агромониторинга
    \item Проектирование и эксплуатация агророботов (на примере AIDA-T)
    \item Анализ данных и принятие решений в точном земледелии
\end{itemize}

\subsection{Программы ДПО}
\textbf{Курс:} ``Современные методы автоматизации и роботизации в тепличном хозяйстве''

\textbf{Формат:} 72 ак. часа, включая практические занятия на симуляторе и демонстрацию работы робота

\subsection{Открытая образовательная платформа}
\begin{itemize}
    \item Открытый датасет изображений с размеченными заболеваниями томатов
    \item Симулятор теплицы и робота AIDA-T в Gazebo/ROS
    \item МООК по основам агроробототехники
\end{itemize}

\section{План реализации проекта}

\textbf{Этап 1 (1-6 мес.):} Концептуальное проектирование, разработка алгоритмов ИИ, создание 3D-моделей

\textbf{Этап 2 (7-12 мес.):} Разработка аппаратной платформы, изготовление гибридной ходовой системы и камерной мачты

\textbf{Этап 3 (13-18 мес.):} Создание программного обеспечения, интеграция всех подсистем

\textbf{Этап 4 (19-24 мес.):} Тестирование в реальных условиях, валидация, подготовка к коммерциализации

\textbf{Результат:} Полностью функционирующий прототип системы AIDA-T уровня TRL 7-8, готовый к демонстрации заказчикам и инвесторам, с подтвержденными техническими характеристиками и полным комплектом документации для последующей коммерциализации.

\end{document}
