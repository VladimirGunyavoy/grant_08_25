\documentclass[12pt,a4paper]{article}
\usepackage[utf8]{inputenc}
\usepackage[russian]{babel}
\usepackage{geometry}
\usepackage{graphicx}
\usepackage{amsmath}
\usepackage{amsfonts}
\usepackage{amssymb}
\usepackage{booktabs}
\usepackage{array}
\usepackage{longtable}
\usepackage{multirow}
\usepackage{wrapfig}
\usepackage{float}
\usepackage{colortbl}
\usepackage{pdflscape}
% \usepackage{tabu}  % Удален - не поддерживается в новых версиях
% \usepackage{threeparttable}  % Удален - может отсутствовать
% \usepackage{threeparttablex}  % Удален - может отсутствовать
% \usepackage{makecell}  % Удален - может отсутствовать
\usepackage{xcolor}
\usepackage{hyperref}
\usepackage{fancyhdr}
\usepackage{titlesec}
\usepackage{enumitem}
\usepackage{setspace}
\usepackage{listings}
\usepackage{xcolor}

% Настройки страницы
\geometry{left=2.5cm,right=2.5cm,top=2.5cm,bottom=2.5cm}

% Настройки гиперссылок
\hypersetup{
    colorlinks=true,
    linkcolor=blue,
    filecolor=magenta,      
    urlcolor=cyan,
    citecolor=green
}

% Настройки заголовков
\titleformat{\section}{\Large\bfseries}{\thesection}{1em}{}
\titleformat{\subsection}{\large\bfseries}{\thesubsection}{1em}{}
\titleformat{\subsubsection}{\normalsize\bfseries}{\thesubsubsection}{1em}{}

% Настройки списков
\setlist[itemize]{leftmargin=*}
\setlist[enumerate]{leftmargin=*}

% Настройки межстрочного интервала
\onehalfspacing

% Настройки колонтитулов
\pagestyle{fancy}
\fancyhf{}
\fancyhead[L]{\small Описание результатов исследований AIDA-T}
\fancyhead[R]{\small \thepage}
\renewcommand{\headrulewidth}{0.4pt}

\begin{document}

% Титульная страница
\begin{titlepage}
    \centering
    \vspace*{2cm}
    
    {\Huge\bfseries Описание предполагаемых результатов исследований, их научной и практической ценности, а также проекта образовательных программ\par}
    
    \vspace{2cm}
    
    {\Large\bfseries Проект: AIDA-T\par}
    {\large (Agrobotic Intelligent Data Analyzer for Tomatoes)\par}
    
    \vspace{1cm}
    
    {\large\bfseries АИДА-Т\par}
    {\large (Агроробототехнический Интеллектуальный Анализатор Данных для Томатов)\par}
    
    \vspace{2cm}
    
    \begin{tabular}{ll}
        \textbf{Руководитель:} & Осиненко Павел Валерьевич \\
        \textbf{Сроки реализации:} & 2024-2026 гг. (24 месяца) \\
        \textbf{Бюджет проекта:} & ~50 млн рублей
    \end{tabular}
    
    \vfill
    
    {\large \today\par}
\end{titlepage}

% Оглавление
\tableofcontents
\newpage

\section{Краткое описание проекта}

Проект АИДА-Т направлен на создание автономной робототехнической системы для интеллектуального мониторинга и диагностики томатов в промышленных теплицах. Основная цель — разработка мобильного робота с системой компьютерного зрения и ИИ, способного автономно перемещаться по теплице, выявлять заболевания растений на ранних стадиях, оценивать урожайность и предоставлять точную аналитическую информацию.

\textbf{Ключевые особенности:}
\begin{itemize}
    \item Гибридная система передвижения (по полу и по рельсам)
    \item Телескопическая камерная мачта с активной стабилизацией
    \item Нейросетевые алгоритмы диагностики заболеваний с точностью 86-87\%
    \item Гибридный метод оценки урожайности с погрешностью не более 15\%
    \item Полная автономность работы до 12 часов
    \item Веб-интерфейс для удаленного мониторинга
\end{itemize}

\section{Предполагаемые результаты исследований}

\subsection{Аппаратная платформа}
\begin{itemize}
    \item \textbf{Гибридная мобильная платформа} с возможностью движения по бетонным покрытиям и рельсовым путям
    \item \textbf{Система технического зрения} с телескопической камерной мачтой (0.1–3.0 м) и активной стабилизацией
    \item \textbf{Интегрированный вычислительный блок} промышленного класса для обработки данных в реальном времени
    \item \textbf{Система энергоснабжения} на базе Li-ion аккумуляторов с автономностью 12+ часов
\end{itemize}

\subsection{Программный комплекс}
\begin{itemize}
    \item \textbf{CNN для диагностики заболеваний} томатов с точностью 86-87\%
    \item \textbf{Гибридный метод оценки урожайности} (RANSAC + PointNet) с погрешностью не более 15\%
    \item \textbf{Система автономной навигации} на базе ROS2 с динамическим переключением режимов
    \item \textbf{Веб-интерфейс} для удаленного мониторинга и управления
\end{itemize}

\subsection{Документация и ИС}
\begin{itemize}
    \item Конструкторская и программная документация
    \item Программы и методики испытаний
    \item 2 заявки на полезные модели
    \item 1 регистрация программы для ЭВМ
\end{itemize}

\section{Научная и практическая ценность}

\subsection{Научная ценность}
Научная новизна проекта заключается в решении междисциплинарных задач на стыке робототехники, компьютерного зрения и агротехнологий:

\begin{itemize}
    \item \textbf{Робототехника:} Концепция гибридной ходовой системы для адаптации к различным поверхностям в теплице
    \item \textbf{Компьютерное зрение:} Специализированная CNN-архитектура для диагностики заболеваний в сложных условиях освещения
    \item \textbf{Системы управления:} Алгоритм автономной навигации с динамическим переключением моделей управления
\end{itemize}

Результаты могут быть опубликованы в журналах Q1/Q2 и представлены на конференциях IEEE IROS, ICRA, AAAI.

\subsection{Практическая ценность}
\subsubsection{Экономический эффект:}
\begin{itemize}
    \item Снижение трудозатрат до 70\%
    \item Повышение урожайности на 15-20\% за счет раннего обнаружения заболеваний
    \item Снижение расхода средств защиты растений на 30-40\%
    \item Период окупаемости 2.5-3 года
\end{itemize}

\subsubsection{Дополнительные преимущества:}
\begin{itemize}
    \item Полностью российская разработка, превосходящая зарубежные аналоги
    \item Решение проблемы дефицита квалифицированных агрономов
    \item Повышение качества продукции
    \item Масштабируемость для других культур и операций
\end{itemize}

\subsection{Команда проекта}
\textbf{Руководитель:} Осиненко Павел Валерьевич, к.т.н.

\textbf{Состав:} 3 доктора наук, 5 кандидатов наук, 8 инженеров-разработчиков, 2 агронома-консультанта

\textbf{Партнеры:} Сколтех, МФТИ, ведущие тепличные комплексы России

\section{Проект образовательных программ}

\subsection{Магистерская программа}
\textbf{Название:} ``Интеллектуальная робототехника в агропромышленном комплексе''

\textbf{Ключевые курсы:}
\begin{itemize}
    \item Мобильная робототехника и навигационные системы в сельском хозяйстве
    \item Компьютерное зрение и машинное обучение для агромониторинга
    \item Проектирование и эксплуатация агророботов (на примере AIDA-T)
    \item Анализ данных и принятие решений в точном земледелии
\end{itemize}

\subsection{Программы ДПО}
\textbf{Курс:} ``Современные методы автоматизации и роботизации в тепличном хозяйстве''

\textbf{Формат:} 72 ак. часа, включая практические занятия на симуляторе и демонстрацию работы робота

\subsection{Открытая образовательная платформа}
\begin{itemize}
    \item Открытый датасет изображений с размеченными заболеваниями томатов
    \item Симулятор теплицы и робота AIDA-T в Gazebo/ROS
    \item МООК по основам агроробототехники
\end{itemize}

\section{План реализации проекта}

\textbf{Этапы:} 1-6 мес. (концептуальное проектирование), 7-12 мес. (аппаратная платформа), 13-18 мес. (ПО), 19-24 мес. (тестирование и валидация)

\textbf{Результат:} Полностью функционирующий прототип системы AIDA-T, готовый к демонстрации заказчикам и инвесторам.

\end{document}
