\documentclass[12pt,a4paper]{article}
\usepackage[utf8]{inputenc}
\usepackage[russian]{babel}
\usepackage{geometry}
\usepackage{graphicx}
\usepackage{amsmath}
\usepackage{amsfonts}
\usepackage{amssymb}
\usepackage{booktabs}
\usepackage{array}
\usepackage{longtable}
\usepackage{multirow}
\usepackage{wrapfig}
\usepackage{float}
\usepackage{colortbl}
\usepackage{pdflscape}
% \usepackage{tabu}  % Удален - не поддерживается в новых версиях
% \usepackage{threeparttable}  % Удален - может отсутствовать
% \usepackage{threeparttablex}  % Удален - может отсутствовать
% \usepackage{makecell}  % Удален - может отсутствовать
\usepackage{xcolor}
\usepackage{hyperref}
\usepackage{fancyhdr}
\usepackage{titlesec}
\usepackage{enumitem}
\usepackage{setspace}
\usepackage{listings}
\usepackage{xcolor}

% Настройки страницы
\geometry{left=2.5cm,right=2.5cm,top=2.5cm,bottom=2.5cm}

% Настройки гиперссылок
\hypersetup{
    colorlinks=true,
    linkcolor=blue,
    filecolor=magenta,      
    urlcolor=cyan,
    citecolor=green
}

% Настройки заголовков
\titleformat{\section}{\Large\bfseries}{\thesection}{1em}{}
\titleformat{\subsection}{\large\bfseries}{\thesubsection}{1em}{}
\titleformat{\subsubsection}{\normalsize\bfseries}{\thesubsubsection}{1em}{}

% Настройки списков
\setlist[itemize]{leftmargin=*}
\setlist[enumerate]{leftmargin=*}

% Настройки межстрочного интервала
\onehalfspacing

% Настройки колонтитулов
\pagestyle{fancy}
\fancyhf{}
\fancyhead[L]{\small Описание результатов исследований AIDA-T}
\fancyhead[R]{\small \thepage}
\renewcommand{\headrulewidth}{0.4pt}

\begin{document}

% Титульная страница
\begin{titlepage}
    \centering
    \vspace*{2cm}
    
    {\Huge\bfseries Описание предполагаемых результатов исследований, их научной и практической ценности, а также проекта образовательных программ\par}
    
    \vspace{2cm}
    
    {\Large\bfseries Проект: AIDA-T\par}
    {\large (Agrobotic Intelligent Data Analyzer for Tomatoes)\par}
    
    \vspace{1cm}
    
    {\large\bfseries АИДА-Т\par}
    {\large (Агроробототехнический Интеллектуальный Анализатор Данных для Томатов)\par}
    
    \vspace{2cm}
    
    \begin{tabular}{ll}
        \textbf{Руководитель:} & Осиненко Павел Валерьевич \\
        \textbf{Сроки реализации:} & 2024-2026 гг. (24 месяца) \\
        \textbf{Бюджет проекта:} & ~50 млн рублей
    \end{tabular}
    
    \vfill
    
    {\large \today\par}
\end{titlepage}

% Оглавление
\tableofcontents
\newpage

\section{Краткое описание проекта}

Проект АИДА-Т направлен на создание передовой автономной робототехнической системы для интеллектуального мониторинга и диагностики томатов в промышленных теплицах. Основная цель проекта — разработка и создание функционирующего прототипа мобильного робота, оснащенного системой компьютерного зрения и искусственного интеллекта, способного автономно перемещаться по теплице, выявлять заболевания растений на ранних стадиях, оценивать урожайность и предоставлять агрономам точную аналитическую информацию для принятия обоснованных решений.

\textbf{Актуальность и потребность рынка:}
Рынок роботизации в АПК России растет на 25-30\% в год, достигнув 15 млрд рублей в 2024 году. Автоматизация позволяет снизить затраты на агрономический контроль до 70\%, что критично в условиях дефицита квалифицированных агрономов и необходимости повышения урожайности для обеспечения продовольственной безопасности.

\textbf{Ключевые особенности системы АИДА-Т:}
\begin{itemize}
    \item Уникальная гибридная система передвижения (меканум-колеса для бетонных покрытий + рельсовые колеса для движения между рядами)
    \item Телескопическая камерная мачта с активной стабилизацией (диапазон 0.1-3.0 м)
    \item Специализированные CNN-алгоритмы диагностики заболеваний с точностью 86-87\%
    \item Гибридный метод оценки урожайности (RANSAC + PointNet) с погрешностью не более 15\%
    \item Полная автономность работы до 12 часов без участия человека
    \item Веб-интерфейс для удаленного мониторинга и управления
    \item Адаптация к российским условиям эксплуатации и полное импортозамещение
\end{itemize}

\textbf{Целевые потребители:}
\begin{itemize}
    \item Крупные агрохолдинги и тепличные комплексы (площадью свыше 5 га) — 150 предприятий в России
    \item Средние тепличные хозяйства (5-10 га) — 300 предприятий
    \item Научно-исследовательские институты в области растениеводства — 50 организаций
    \item Образовательные учреждения аграрного профиля — 200 вузов и техникумов
\end{itemize}

\textbf{Планируемые технические параметры:}
\begin{itemize}
    \item Точность диагностики заболеваний: 86-87\% (превосходит существующие решения на 10-15\%)
    \item Точность оценки объема плодов: погрешность не более 15\%
    \item Точность позиционирования при съемке: ±1,5 мм на пиксель
    \item Время автономной работы: не менее 12 часов
    \item Стоимость системы: 15-20 млн рублей (период окупаемости 2.5-3 года)
\end{itemize}

\section{Научно-техническая новизна и методы решения задач}

\subsection{Научно-техническая новизна проекта}

Научно-техническая новизна проекта AIDA-T заключается в комплексном решении задач автономного мониторинга растений в условиях промышленных теплиц через интеграцию инновационных аппаратных и программных решений.

\subsubsection{Техническая новизна аппаратной части}
\begin{itemize}
    \item \textbf{Гибридная ходовая система} — впервые предложена комбинация меканум-колес для движения по бетонному покрытию и специальных нейлоновых рельсовых колес для автономного взбирания и движения по рельсовым путям между рядами культур без использования дополнительных механических приспособлений
    \item \textbf{Адаптивная амортизация с компенсацией вибраций} — разработанная система стабилизации обеспечивает точность позиционирования камерной мачты ±1.5 мм на пиксель даже при движении по рельсам, что превосходит точность существующих мобильных систем мониторинга растений в 2-3 раза
    \item \textbf{Камерная мачта повышенной продольной жесткости} с регулировкой высоты в диапазоне 0.1-3.0 метра и системой слияния кадров с пересекающимися углами обзора для повышения точности получаемых данных
\end{itemize}

\subsubsection{Программная новизна}
\begin{itemize}
    \item \textbf{Гибридный подход к оценке объема плодов} — впервые реализована параллельная обработка данных двумя независимыми методами (алгоритм RANSAC и нейросетевой подход PointNet) с последующим сравнительным анализом результатов, что повышает надежность оценки урожайности
    \item \textbf{Специализированная CNN-архитектура} для диагностики заболеваний томатов в условиях переменного освещения теплиц, достигающая точности 0.86-0.87 при детекции мучнистой росы против 0.70-0.80 у существующих решений
    \item \textbf{Алгоритм автономной навигации} в структурированной среде теплицы с динамическим переключением между режимами движения (бетон/рельсы) на основе компьютерного зрения
\end{itemize}

\subsubsection{Системная новизна}
Впервые предложена полностью автономная система круглосуточного мониторинга с возможностью работы без участия агрономов в течение всего цикла диагностики, что кардинально отличается от существующих решений, требующих постоянного присутствия оператора или работающих по заданным маршрутам.

\subsection{Методы и способы решения поставленных задач}

\subsubsection{Задача 1: Создание гибридной мобильной платформы}
\textbf{Метод решения:} Разработка и интеграция двухрежимной ходовой системы на базе:
\begin{itemize}
    \item Меканум-колес с бесщеточными мотор-редукторами для всенаправленного движения по бетонным дорожкам
    \item Нейлоновых рельсовых колес для движения по технологическим рельсам между рядами растений
    \item Системы адаптивной амортизации на базе колесного модуля с элементами амортизации
\end{itemize}

\textbf{Новизна подхода:} В отличие от существующих систем с фиксированным типом передвижения, предлагаемое решение обеспечивает адаптивность к различным участкам теплицы без необходимости модификации инфраструктуры.

\subsubsection{Задача 2: Обеспечение высокоточного позиционирования}
\textbf{Метод решения:}
\begin{itemize}
    \item Проектирование камерной мачты с повышенной продольной жесткостью из композитных материалов
    \item Применение промышленных Ethernet-камер с глобальным затвором для исключения эффекта ``rolling shutter''
    \item Реализация системы слияния кадров (image stitching) с пересекающимися углами обзора
\end{itemize}

\subsubsection{Задача 3: Создание алгоритмов компьютерного зрения}
\textbf{Метод решения диагностики заболеваний:}
\begin{itemize}
    \item Разработка специализированной CNN-архитектуры на базе модифицированной ResNet с дополнительными блоками внимания (attention mechanism)
    \item Создание расширенного датасета изображений томатов с различными стадиями заболеваний
    \item Применение техник аугментации данных для моделирования различных условий освещения
\end{itemize}

\textbf{Метод решения оценки объема плодов:}
\begin{itemize}
    \item Параллельная обработка стереоизображений двумя независимыми алгоритмами:
    \begin{enumerate}
        \item Классический геометрический подход на основе RANSAC для выделения контуров
        \item Нейросетевой подход PointNet для прямой оценки объема по облаку точек
    \end{enumerate}
    \item Сравнительный анализ результатов двух методов для повышения надежности оценки
\end{itemize}

\section{Имеющийся научно-технический задел}

\subsection{Теоретический задел}
Команда проекта обладает глубокими компетенциями в области:
\begin{itemize}
    \item \textbf{Компьютерного зрения и машинного обучения:} 5-летний опыт разработки CNN-архитектур для задач детекции и классификации в агропромышленности
    \item \textbf{Робототехники:} 5-летний опыт создания автономных мобильных платформ для промышленного применения
    \item \textbf{Агротехнологий:} экспертиза в области выращивания томатов в защищенном грунте
\end{itemize}

\subsection{Экспериментальный задел}
\begin{itemize}
    \item \textbf{Предварительные исследования CNN:} создан базовый датасет из 15,000 изображений томатов с разметкой заболеваний, достигнута точность детекции мучнистой росы 0.83 на лабораторных образцах
    \item \textbf{Тестирование алгоритмов оценки объема:} реализованы и протестированы базовые версии RANSAC и PointNet алгоритмов, достигнута ошибка оценки объема ~12\% на тестовых образцах
    \item \textbf{Прототипирование ходовой системы:} изготовлен и испытан макет меканум-платформы грузоподъемностью до 100 кг
\end{itemize}

\subsection{Технический задел}
\begin{itemize}
    \item \textbf{Аппаратная база:} определены и протестированы основные компоненты системы
    \item \textbf{Программная архитектура:} спроектирована модульная архитектура системы на базе ROS2
    \item \textbf{3D-моделирование:} создана 3D-модель будущего прототипа с проработкой основных узлов
\end{itemize}

\section{Конкурентный анализ и преимущества}

\subsection{Анализ конкурентов}
\textbf{Прямые конкуренты:}
\begin{itemize}
    \item Bosch DeepField Robotics (Германия) — мобильные роботы для теплиц
    \item Iron Ox (США) — автономные системы мониторинга растений
    \item Harvest CROO Robotics (США) — роботизированные системы для сельского хозяйства
    \item ООО ``Агроробот'' (Россия) — системы автоматизации теплиц
\end{itemize}

\textbf{Косвенные конкуренты:}
\begin{itemize}
    \item Стационарные системы видеоаналитики (Netafim, Philips GrowWise)
    \item Ручной мониторинг агрономами
    \item Дроны для мониторинга теплиц (Sentera, DroneDeploy)
\end{itemize}

\subsection{Сравнительный анализ технических характеристик}

\begin{table}[h]
\centering
\begin{tabular}{|p{4cm}|p{2.5cm}|p{2.5cm}|p{2.5cm}|p{2.5cm}|}
\hline
\textbf{Параметр} & \textbf{AIDA-T} & \textbf{Bosch DeepField} & \textbf{Iron Ox} & \textbf{Агроробот} \\
\hline
Точность диагностики заболеваний & 86-87\% & 75-80\% & 70-75\% & 65-70\% \\
\hline
Ошибка оценки объема плодов & ≤15\% & 15-20\% & 12-15\% & Не заявлена \\
\hline
Точность позиционирования & ±1.5 мм на пиксель & ±5 мм на пиксель & ±3 мм на пиксель & ±10 мм на пиксель \\
\hline
Тип ходовой системы & Гибридная (колеса+рельсы) & Колесная стандартная & Рельсовая фиксированная & Стационарная \\
\hline
Стоимость системы & 15-20 млн руб & \$150-200 тыс & \$120-180 тыс & 1.5-2.0 млн руб \\
\hline
\end{tabular}
\caption{Сравнение AIDA-T с основными конкурентами}
\end{table}

\subsection{Ключевые конкурентные преимущества AIDA-T}
\begin{itemize}
    \item \textbf{Высокая точность диагностики:} 86-87\% против 70-80\% у конкурентов
    \item \textbf{Уникальная гибридная ходовая система:} единственное решение, способное работать как на бетонном покрытии, так и на рельсовых путях
    \item \textbf{Передовые алгоритмы оценки урожайности:} параллельное использование методов RANSAC и PointNet
    \item \textbf{Импортозамещение:} полностью российская разработка с локализацией производства
    \item \textbf{Адаптация к российским условиям:} специально разработано для российских теплиц
\end{itemize}

\section{Анализ рынка и экономическое обоснование}

\subsection{Объем и динамика рынка}

\textbf{Российский рынок защищенного грунта:}
\begin{itemize}
    \item Объем рынка в 2024 году: 280 млрд рублей
    \item Среднегодовой рост: 12-15\%
    \item Прогноз на 2030 год: 450-500 млрд рублей
\end{itemize}

\textbf{Рынок роботизации в АПК России:}
\begin{itemize}
    \item Объем в 2024 году: 15 млрд рублей
    \item Среднегодовой рост: 25-30\%
    \item Прогноз на 2030 год: 65-75 млрд рублей
\end{itemize}

\textbf{Доступный рынок для AIDA-T (SAM):}
\begin{itemize}
    \item 2025 год: 2.1 млрд рублей
    \item 2030 год: 4.8 млрд рублей
    \item Целевая доля рынка к 2030 году: 3-5\% (150-250 млн рублей)
\end{itemize}

\subsection{Экономическое обоснование}

\textbf{Экономические преимущества для потребителя:}
\begin{itemize}
    \item Снижение трудозатрат на визуальный контроль растений до 70\%
    \item Повышение урожайности за счет раннего выявления заболеваний на 15-20\%
    \item Снижение расхода средств защиты растений на 30-40\%
    \item Период окупаемости системы: 2.5-3 года
\end{itemize}

\textbf{Финансовая модель проекта:}

\begin{table}[h]
\centering
\begin{tabular}{|l|c|c|c|c|c|}
\hline
\textbf{Год} & \textbf{2025} & \textbf{2026} & \textbf{2027} & \textbf{2028} & \textbf{2029} \\
\hline
Количество систем & 3 & 8 & 15 & 25 & 40 \\
\hline
Выручка (млн руб) & 53 & 140 & 263 & 438 & 700 \\
\hline
Валовая прибыль (млн руб) & 18 & 49 & 92 & 153 & 245 \\
\hline
Рентабельность (\%) & 34\% & 35\% & 35\% & 35\% & 35\% \\
\hline
\end{tabular}
\caption{Прогноз финансовых показателей}
\end{table}

\section{Бизнес-модель и стратегия коммерциализации}

\subsection{Бизнес-модель}
\textbf{Основная модель монетизации:} B2B продажи комплексных решений с сервисным обслуживанием

\textbf{Структура доходов:}
\begin{itemize}
    \item \textbf{Продажа оборудования (70\% выручки):} стоимость системы AIDA-T 15-20 млн рублей, маржинальность 40-45\%
    \item \textbf{Сервисное обслуживание (25\% выручки):} годовое обслуживание 1.2-1.6 млн рублей, маржинальность 60-70\%
    \item \textbf{Дополнительные услуги (5\% выручки):} обучение персонала, консалтинг, маржинальность 70-80\%
\end{itemize}

\subsection{Стратегия продвижения на рынок}

\textbf{Этап 1 (2025-2026): Выход на рынок}
\begin{itemize}
    \item Пилотные проекты с 3-5 ведущими агрохолдингами
    \item Участие в профильных выставках (Агрорусс, ГолденОсень, Агропродмаш)
    \item Публикации в отраслевых изданиях и научных журналах
    \item Создание демонстрационной площадки
\end{itemize}

\textbf{Этап 2 (2026-2028): Масштабирование}
\begin{itemize}
    \item Расширение линейки продуктов (адаптация для огурцов, перца)
    \item Развитие дилерской сети в регионах
    \item Партнерства с системными интеграторами
    \item Выход на рынки СНГ
\end{itemize}

\textbf{Каналы продаж:}
\begin{itemize}
    \item Прямые продажи (60\%): работа с топ-20 агрохолдингами России
    \item Партнерская сеть (30\%): дилеры в регионах (15-20 компаний к 2028 году)
    \item Интернет-маркетинг (10\%): корпоративный сайт, контент-маркетинг
\end{itemize}

\section{Предполагаемые результаты исследований}

\subsection{Аппаратная платформа}
\begin{itemize}
    \item \textbf{Гибридная мобильная платформа} с возможностью движения по бетонным покрытиям и рельсовым путям
    \item \textbf{Система технического зрения} с телескопической камерной мачтой (0.1–3.0 м) и активной стабилизацией
    \item \textbf{Интегрированный вычислительный блок} промышленного класса для обработки данных в реальном времени
    \item \textbf{Система энергоснабжения} на базе Li-ion аккумуляторов с автономностью 12+ часов
\end{itemize}

\subsection{Программный комплекс}
\begin{itemize}
    \item \textbf{CNN для диагностики заболеваний} томатов с точностью 86-87\%
    \item \textbf{Гибридный метод оценки урожайности} (RANSAC + PointNet) с погрешностью не более 15\%
    \item \textbf{Система автономной навигации} на базе ROS2 с динамическим переключением режимов
    \item \textbf{Веб-интерфейс} для удаленного мониторинга и управления
\end{itemize}

\subsection{Документация и ИС}
\begin{itemize}
    \item Конструкторская и программная документация
    \item Программы и методики испытаний
    \item 2 заявки на полезные модели
    \item 1 регистрация программы для ЭВМ
\end{itemize}

\section{Научная и практическая ценность}

\subsection{Научная ценность}
Научная новизна проекта заключается в решении междисциплинарных задач на стыке робототехники, компьютерного зрения и агротехнологий:

\begin{itemize}
    \item \textbf{Робототехника:} Концепция гибридной ходовой системы для адаптации к различным поверхностям в теплице
    \item \textbf{Компьютерное зрение:} Специализированная CNN-архитектура для диагностики заболеваний в сложных условиях освещения
    \item \textbf{Системы управления:} Алгоритм автономной навигации с динамическим переключением моделей управления
\end{itemize}

Результаты могут быть опубликованы в журналах Q1/Q2 и представлены на конференциях IEEE IROS, ICRA, AAAI.

\subsection{Практическая ценность}
\subsubsection{Экономический эффект:}
\begin{itemize}
    \item Снижение трудозатрат до 70\%
    \item Повышение урожайности на 15-20\% за счет раннего обнаружения заболеваний
    \item Снижение расхода средств защиты растений на 30-40\%
    \item Период окупаемости 2.5-3 года
\end{itemize}

\subsubsection{Дополнительные преимущества:}
\begin{itemize}
    \item Полностью российская разработка, превосходящая зарубежные аналоги
    \item Решение проблемы дефицита квалифицированных агрономов
    \item Повышение качества продукции
    \item Масштабируемость для других культур и операций
\end{itemize}

\subsection{Команда проекта}

\subsubsection{Научный руководитель}
\textbf{Осиненко Павел Валерьевич} — доктор технических наук, доцент кафедры инженерных систем Сколковского института науки и технологий

\textbf{Квалификация и опыт:}
\begin{itemize}
    \item Доктор технических наук в области систем управления и робототехники
    \item Научный руководитель PhD программы по инженерным системам Сколтеха
    \item Руководитель множественных исследовательских проектов в области автономных систем
    \item Автор научных публикаций в области оптимального управления и робототехники
    \item Эксперт в области математического моделирования и оптимизации
\end{itemize}

\subsubsection{Ключевые участники проекта}

\begin{table}[h]
\centering
\begin{tabular}{|p{3.5cm}|p{3cm}|p{4cm}|p{4.5cm}|}
\hline
\textbf{ФИО} & \textbf{Роль} & \textbf{Квалификация} & \textbf{Опыт} \\
\hline
Давиденко Сергей Александрович & Руководитель проекта, технический директор & Аспирант Сколтеха, Chief Robotics Engineer Сбербанк & Победитель гранта УМНИК-2021, 10+ лет в робототехнике, публикации в IEEE \\
\hline
Рякин Илья Сергеевич & Главный разработчик CV & Аспирант, 6 лет опыта в компьютерном зрении & Руководство CV компонентов в предыдущих проектах, научные публикации \\
\hline
Осокин Илья Олегович & Научный консультант & Аспирант, преподаватель МФТИ & 6 лет преподавания робототехники, 5+ международных публикаций \\
\hline
Гунявой Владимир & Математический инженер & Математический инженер, Skoltech & Разработка алгоритмов для робототехники, работа в Topcon Lab \\
\hline
\end{tabular}
\caption{Состав команды проекта}
\end{table}

\subsubsection{Опыт команды в реализации подобных проектов}
\begin{itemize}
    \item Успешная реализация проекта ``Мобильный логистический робот с интеллектуальным управлением'' (грант УМНИК-2021)
    \item Руководство модернизацией платформы Cobot Magic в Сбербанк
    \item 6 лет практического опыта разработки алгоритмов компьютерного зрения
    \item Регистрация 2 программ для ЭВМ, подача материалов на международные конференции (IEEE SMC, IROS)
\end{itemize}

\subsubsection{Партнеры проекта}
\begin{itemize}
    \item \textbf{Научные партнеры:} Сколковский институт науки и технологий, Московский физико-технический институт
    \item \textbf{Промышленные партнеры:} Ведущие тепличные комплексы России для тестирования и валидации
    \item \textbf{Технологические партнеры:} Компании по металлообработке, производители компонентов
\end{itemize}

\section{Планируемая интеллектуальная собственность}

\subsection{Объекты интеллектуальной собственности}

\textbf{Планируемые полезные модели:}
\begin{itemize}
    \item ``Гибридная ходовая система мобильного робота для теплиц'' — подача заявки до 6 месяца проекта
    \item ``Камерная мачта с адаптивной стабилизацией для мобильных роботов'' — подача заявки до 8 месяца проекта
\end{itemize}

\textbf{Программы для ЭВМ:}
\begin{itemize}
    \item ``Программный комплекс диагностики заболеваний растений на основе CNN'' — регистрация до 10 месяца проекта
    \item ``Система автономной навигации мобильного робота в теплице'' — регистрация до 11 месяца проекта
\end{itemize}

\textbf{Перспективные патенты на изобретения:}
\begin{itemize}
    \item ``Способ оценки объема плодов с использованием гибридных алгоритмов'' — подача заявки в течение 6 месяцев после завершения НИОКР
    \item ``Система мониторинга растений с многоракурсным сканированием'' — подача заявки в течение 12 месяцев после завершения НИОКР
\end{itemize}

\subsection{Мероприятия по патентным исследованиям}
\begin{itemize}
    \item Проведение патентного поиска по тематике робототехники для сельского хозяйства до 3 месяца проекта
    \item Анализ патентной чистоты разрабатываемых технических решений до 6 месяца проекта
    \item Мониторинг патентной активности конкурентов в течение всего периода НИОКР
\end{itemize}

\section{Анализ рисков проекта}

\subsection{Технические риски}

\begin{table}[h]
\centering
\begin{tabular}{|p{4cm}|p{2cm}|p{2cm}|p{6cm}|}
\hline
\textbf{Риск} & \textbf{Вероятность} & \textbf{Влияние} & \textbf{Меры по снижению} \\
\hline
Невыполнение заявленных характеристик точности & Средняя & Высокое & Дополнительные испытания, привлечение экспертов, использование предобученных моделей \\
\hline
Проблемы интеграции аппаратной и программной частей & Низкая & Среднее & Поэтапное тестирование, модульная архитектура \\
\hline
Сложности с адаптацией к различным типам теплиц & Средняя & Среднее & Тестирование в различных условиях, гибкая настройка параметров \\
\hline
\end{tabular}
\caption{Технические риски проекта}
\end{table}

\subsection{Рыночные и коммерческие риски}

\begin{table}[h]
\centering
\begin{tabular}{|p{4cm}|p{2cm}|p{2cm}|p{6cm}|}
\hline
\textbf{Риск} & \textbf{Вероятность} & \textbf{Влияние} & \textbf{Меры по снижению} \\
\hline
Снижение спроса на автоматизацию в АПК & Низкая & Высокое & Диверсификация на другие культуры и отрасли \\
\hline
Появление сильных конкурентов & Средняя & Среднее & Защита ИС, постоянные инновации \\
\hline
Превышение бюджета НИОКР & Средняя & Среднее & Детальное планирование, резерв 15\% \\
\hline
Задержка привлечения инвестиций & Высокая & Высокое & Множественные источники финансирования \\
\hline
\end{tabular}
\caption{Рыночные и коммерческие риски}
\end{table}

\subsection{Регуляторные риски}
\begin{itemize}
    \item \textbf{Изменение требований к робототехнике в АПК:} мониторинг законодательства, участие в отраслевых ассоциациях
    \item \textbf{Сложности с сертификацией:} работа с профильными организациями с начала проекта
    \item \textbf{Импортные ограничения на компоненты:} поиск российских поставщиков, создание запасов
\end{itemize}

\section{Проект образовательных программ}

\subsection{Магистерская программа}
\textbf{Название:} ``Интеллектуальная робототехника в агропромышленном комплексе''

\textbf{Ключевые курсы:}
\begin{itemize}
    \item Мобильная робототехника и навигационные системы в сельском хозяйстве
    \item Компьютерное зрение и машинное обучение для агромониторинга
    \item Проектирование и эксплуатация агророботов (на примере AIDA-T)
    \item Анализ данных и принятие решений в точном земледелии
\end{itemize}

\subsection{Программы ДПО}
\textbf{Курс:} ``Современные методы автоматизации и роботизации в тепличном хозяйстве''

\textbf{Формат:} 72 ак. часа, включая практические занятия на симуляторе и демонстрацию работы робота

\subsection{Открытая образовательная платформа}
\begin{itemize}
    \item Открытый датасет изображений с размеченными заболеваниями томатов
    \item Симулятор теплицы и робота AIDA-T в Gazebo/ROS
    \item МООК по основам агроробототехники
\end{itemize}

\section{Детальный план реализации проекта}

\subsection{Общая структура проекта}
Проект AIDA-T рассчитан на 24 месяца и разделен на 4 основных этапа по 6 месяцев каждый. Каждый этап имеет четкие цели, задачи и критерии завершения.

\subsection{Этап 1 (месяцы 1-6): Концептуальное проектирование и разработка алгоритмов}

\textbf{Основные задачи:}
\begin{itemize}
    \item Детальное техническое задание на систему AIDA-T
    \item Разработка архитектуры программно-аппаратного комплекса
    \item Создание и обучение прототипов алгоритмов машинного обучения
    \item Подготовка и расширение датасета для обучения CNN
    \item Проектирование механических узлов (3D-модели, чертежи)
    \item Проведение патентного поиска и анализа конкурентов
\end{itemize}

\textbf{Ожидаемые результаты:}
\begin{itemize}
    \item Техническое задание и спецификации всех подсистем
    \item Архитектура программной системы на базе ROS2
    \item Обученная CNN-модель с точностью диагностики не менее 83\%
    \item Расширенный датасет (25,000+ изображений с разметкой)
    \item Комплект 3D-моделей и эскизной конструкторской документации
    \item Отчет о патентной чистоте разрабатываемых решений
\end{itemize}

\textbf{Критерии завершения этапа:}
\begin{itemize}
    \item Успешная валидация CNN-модели на тестовом наборе данных
    \item Завершение 3D-моделирования всех основных узлов
    \item Подача заявки на первую полезную модель
    \item Утверждение технического задания научным руководителем
\end{itemize}

\subsection{Этап 2 (месяцы 7-12): Разработка аппаратной платформы}

\textbf{Основные задачи:}
\begin{itemize}
    \item Проектирование и изготовление гибридной ходовой системы
    \item Разработка камерной мачты с системой активной стабилизации
    \item Интеграция вычислительного оборудования и систем связи
    \item Создание системы энергоснабжения с контроллером заряда
    \item Сборка и первичная настройка аппаратной платформы
    \item Тестирование механических узлов и систем управления
\end{itemize}

\textbf{Ожидаемые результаты:}
\begin{itemize}
    \item Функционирующая мобильная платформа с гибридной ходовой системой
    \item Камерная мачта с регулировкой высоты 0.1-3.0 м и стабилизацией ±1.5 мм
    \item Интегрированная система управления на базе промышленного компьютера
    \item Автономная система энергоснабжения на 12+ часов работы
    \item Протоколы испытаний механических узлов
    \item Подача заявки на вторую полезную модель
\end{itemize}

\textbf{Критерии завершения этапа:}
\begin{itemize}
    \item Успешные испытания ходовой системы на тестовом полигоне
    \item Подтверждение точности позиционирования камерной системы
    \item Проверка автономности работы в течение 12 часов
    \item Интеграция всех аппаратных компонентов в единую систему
\end{itemize}

\subsection{Этап 3 (месяцы 13-18): Разработка программного обеспечения}

\textbf{Основные задачи:}
\begin{itemize}
    \item Реализация алгоритмов диагностики заболеваний (CNN)
    \item Создание гибридной системы оценки урожайности (RANSAC + PointNet)
    \item Разработка системы автономной навигации и планирования маршрутов
    \item Создание веб-интерфейса для мониторинга и управления
    \item Интеграция всех программных модулей в единую систему
    \item Оптимизация производительности и отладка
\end{itemize}

\textbf{Ожидаемые результаты:}
\begin{itemize}
    \item Программный комплекс диагностики с точностью 86-87\%
    \item Система оценки урожайности с погрешностью не более 15\%
    \item Автономная навигационная система с поддержкой двух режимов движения
    \item Веб-интерфейс с функциями мониторинга и формирования отчетов
    \item Полная программная документация и инструкции пользователя
    \item Регистрация двух программ для ЭВМ
\end{itemize}

\textbf{Критерии завершения этапа:}
\begin{itemize}
    \item Достижение заявленной точности диагностики заболеваний
    \item Валидация алгоритмов оценки урожайности на тестовых данных
    \item Успешное выполнение автономного цикла мониторинга
    \item Завершение интеграционного тестирования всех подсистем
\end{itemize}

\subsection{Этап 4 (месяцы 19-24): Тестирование и валидация в реальных условиях}

\textbf{Основные задачи:}
\begin{itemize}
    \item Лабораторные испытания полного прототипа системы
    \item Полевые испытания в реальных условиях промышленных теплиц
    \item Оптимизация и доработка системы по результатам испытаний
    \item Подготовка полного комплекта технической документации
    \item Подача заявок на патенты и регистрация результатов ИС
    \item Подготовка к коммерциализации и демонстрации инвесторам
\end{itemize}

\textbf{Ожидаемые результаты:}
\begin{itemize}
    \item Полностью функционирующий прототип системы AIDA-T
    \item Протоколы лабораторных и полевых испытаний
    \item Подтверждение всех заявленных технических характеристик
    \item Полный комплект технической и программной документации
    \item Заявки на 2 патента на изобретения
    \item Готовность к началу серийного производства
\end{itemize}

\textbf{Критерии завершения этапа:}
\begin{itemize}
    \item Успешная работа системы в реальных условиях теплицы
    \item Подтверждение заявленных характеристик точности и автономности
    \item Получение положительных отзывов от тестовых площадок
    \item Завершение подготовки всей отчетной документации
\end{itemize}

\subsection{Контрольные точки и управление рисками}

\textbf{Ключевые контрольные точки:}
\begin{itemize}
    \item Месяц 6: Защита технического задания и демонстрация алгоритмов
    \item Месяц 12: Демонстрация работы аппаратной платформы
    \item Месяц 18: Демонстрация интегрированной системы в лабораторных условиях
    \item Месяц 24: Финальная демонстрация работы в реальных условиях
\end{itemize}

\textbf{Система управления рисками:}
\begin{itemize}
    \item Еженедельные совещания команды проекта
    \item Ежемесячные отчеты о ходе выполнения работ
    \item Резерв времени 15\% на каждом этапе для устранения технических проблем
    \item Параллельная разработка критически важных компонентов
    \item Постоянный мониторинг технологических и рыночных изменений
\end{itemize}

\subsection{Ресурсное обеспечение}

\textbf{Человеческие ресурсы:}
\begin{itemize}
    \item Постоянная команда: 4 специалиста (руководитель проекта, CV-разработчик, консультант, математический инженер)
    \item Привлекаемые специалисты: механический инженер, агроном-консультант (по мере необходимости)
    \item Научное руководство: Осиненко П.В. (Сколтех)
\end{itemize}

\textbf{Материально-техническое обеспечение:}
\begin{itemize}
    \item Лабораторная база Сколковского института науки и технологий
    \item Доступ к тепличному комплексу для полевых испытаний
    \item Производственные мощности для изготовления прототипа
    \item Вычислительные ресурсы для обучения нейронных сетей
\end{itemize}

\textbf{Итоговый результат проекта:}
Полностью функционирующий прототип системы AIDA-T уровня TRL 7-8, готовый к демонстрации потенциальным заказчикам и инвесторам, с подтвержденными техническими характеристиками и полным комплектом документации для последующей коммерциализации.

\end{document}
