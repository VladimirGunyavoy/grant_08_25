\documentclass[12pt,a4paper]{article}
\usepackage[T2A]{fontenc}
\usepackage[utf8]{inputenc}
\usepackage[russian]{babel}
\usepackage{geometry}
\usepackage{graphicx}
\usepackage{amsmath}
\usepackage{amsfonts}
\usepackage{amssymb}
\usepackage{booktabs}
\usepackage{array}
\usepackage{multirow}
\usepackage{xcolor}
\usepackage{hyperref}
\usepackage{bookmark}
\usepackage{fancyhdr}
\usepackage{titlesec}
\usepackage{enumitem}
\usepackage{setspace}
\usepackage[final,protrusion=true,expansion=false,nopatch=footnote]{microtype}
\microtypesetup{protrusion=true,expansion=false}


% Настройки страницы
\geometry{left=2cm,right=2cm,top=2cm,bottom=2cm}

% Настройки гиперссылок
\hypersetup{
    colorlinks=true,
    linkcolor=blue,
    filecolor=magenta,      
    urlcolor=cyan,
    citecolor=green
}

% Настройки заголовков
\titleformat{\section}{\Large\bfseries}{\thesection}{1em}{}[\vspace{0.4em}\titlerule]
\titleformat{\subsection}{\large\bfseries}{\thesubsection}{1em}{}[\vspace{0.25em}\titlerule]
\titleformat{\subsubsection}{\normalsize\bfseries}{\thesubsubsection}{1em}{}[\vspace{0.2em}\titlerule]
\titlespacing*{\section}{0pt}{1.0ex plus .2ex minus .2ex}{0.8ex}
\titlespacing*{\subsection}{0pt}{0.8ex plus .2ex minus .2ex}{0.6ex}
\titlespacing*{\subsubsection}{0pt}{0.6ex plus .2ex minus .2ex}{0.5ex}

% Настройки списков
\setlist[itemize]{leftmargin=*,itemsep=1pt,parsep=0pt}
\setlist[enumerate]{leftmargin=*,itemsep=1pt,parsep=0pt}

% Настройки межстрочного интервала (для укладки в 5 страниц)
\singlespacing%
\frenchspacing
\emergencystretch=2em

% Настройки колонтитулов
\pagestyle{fancy}
\fancyhf{}
\fancyhead[L]{\small AIDA: Исследования и образовательная программа}
\fancyhead[R]{\small \thepage}
\renewcommand{\headrulewidth}{0.4pt}
\setlength{\headheight}{14pt}

% Локальный .bib с ссылками (уникальное имя, чтобы избежать конфликтов)
% Для компиляции используйте внешний local_ai_refs.bib в той же директории

\begin{document}

% Титульная страница
\begin{titlepage}
    \centering
    \vspace*{1cm}
    
    {\Huge\bfseries Лаборатория AIDA: Исследовательско-образовательная программа\par}
    
    \vspace{1.5cm}
    
    {\Large\bfseries Лаборатория AIDA\par}
    {\large (\textit{Artificial Intelligence in Dynamic Action})\par}
    
    \vspace{1.5cm}
    
    \begin{tabular}{ll}
        \textbf{Руководитель:} & Осиненко Павел Валерьевич \\
        \textbf{Институция:} & Сколковский институт науки и технологий (Сколтех)
    \end{tabular}
    
    \vfill
    
    {\large \today\par}
\end{titlepage}

\section{Лаборатория AIDA: цели и подход}

Лаборатория AIDA занимается тем, что делает системы с искусственным интеллектом безопасными и надёжными в реальных условиях. Наша цель на 2025–2030 годы — войти в число ведущих групп России в области современного управления и обучения с подкреплением, узнаваемых на международном уровне. Для этого мы объединяем сильные стороны классической теории управления (модельно-прогнозирующее управление, гарантии стабильности и безопасности) с современным ИИ (в том числе обучением с подкреплением), чтобы принести практическую пользу в робототехнике и других отраслях.

Подход строится на простой идее: самые продвинутые роботы и автономные системы по‑прежнему опираются на проверенные методы управления с формальными гарантиями. ИИ усиливает эти методы там, где модели неполны или среда меняется, но безопасность и предсказуемость должны оставаться в центре. Мы разрабатываем решения, где алгоритмы ИИ работают совместно с проверенными контроллерами и средствами верификации, чтобы «лучшее из двух миров» было доступно инженерам.

В рамках этого подхода на горизонте 2025–2030 годов мы планируем:
\begin{itemize}
    \item развить \textbf{платформу AIDA‑T} (Agrobotic Intelligent Data Analyzer for Tomatoes) от уровня опытного образца с TRL~5 до промышленного решения для автономных теплиц (на базе регистрируемого стартапа «AIDA Robotics»);
    \item довести \textbf{программный комплекс Regelum} — открываемую альтернативу MATLAB для прототипирования продвинутых систем управления и автоматизации с ИИ — до состояния продукта для корпоративных клиентов (лицензирование и индустриальные партнёрства).
\end{itemize}

Мы также ведём образовательную программу, готовящую инженеров и исследователей для автономных систем, и публикуем результаты в ведущих журналах и на конференциях.

\textbf{Избранные публикации:}
\begin{itemize}
    \item \href{https://arxiv.org/abs/2501.02267}{\textit{Towards a constructive framework for control theory}}
    \item \href{https://arxiv.org/abs/2105.07152}{\textit{On stochastic stabilization of sampled systems}}
    \item \href{https://arxiv.org/abs/2205.13409}{\textit{On stochastic stabilization via non-smooth control Lyapunov functions}}
\end{itemize}

\section{О лаборатории}

\textbf{Лаборатория Искусственного интеллекта в динамических системах (AIDA)} --- научно-исследовательская группа Сколтеха, ведущая фундаментальные и прикладные исследования ИИ в динамических системах: безопасное автоматическое управление, обучение с подкреплением с гарантиями, численная надёжность алгоритмов и методы планирования и поддержки принятия решений; внедрение в робототехнике, финтехе и промышленной автоматизации.

\textbf{Миссия:} Обеспечивать безопасное и надёжное применение ИИ в динамических системах через разработку методов с формальными гарантиями и их практическое внедрение.

\subsection*{Основные направления деятельности}
\begin{itemize}
    \item Проектирование и внедрение безопасных систем автоматического управления для критических применений
    \item Инжиниринг RL-решений с гарантиями безопасности (\textit{CALF}) под отраслевые задачи
    \item Развитие и поддержка платформы Regelum: \textit{ML}-ready платформы, \textit{ROS}-интеграция, симуляция и испытательные стенды
    \item НИОКР и консалтинг: робототехника, компьютерное зрение, финтех, промышленная автоматизация
\end{itemize}

\section{Ключевые научные направления}
\begin{itemize}
    \item Теория безопасного обучения с подкреплением: Lyapunov-based \textit{RL} (\textit{CALF}), \textit{constrained MDPs}; цели --- сертификаты стабильности и безопасности
    \item Планирование и поддержка принятия решений в динамических системах: MPC, POMDP, task-and-motion planning; цели --- безопасность и оптимальность
    \item Численная надёжность алгоритмов управления: алгоритмическая неопределённость, интервальный/символьный анализ, верификация вычислений
    \item Восприятие и локализация для автономных систем: компьютерное зрение, навигация, SLAM; цели --- устойчивость к шумам и сбоям
\end{itemize}

\subsection*{Почему это важно}
Индустрии нужны системы, которые не просто «работают в среднем», а сохраняют безопасность и предсказуемость в реальных, шумных и неполных условиях. Даже самые успешные ИИ‑подходы уязвимы к сбоям датчиков, неточностям моделей и ограниченным вычислительным ресурсам. Разрыв между теорией и практикой проявляется в виде нестабильности, неожиданных ошибок оптимизации и трудностей сертификации.

Наш фокус — сократить этот разрыв. Мы объединяем: (i) проверенные методы управления с формальными гарантиями (стабильность, безопасность, соблюдение ограничений), (ii) обучение с подкреплением и планирование для адаптивности и эффективности, (iii) инструменты численной надёжности и верификации, чтобы инженер мог доверять итоговой системе. Такой подход делает ИИ‑решения применимыми там, где ставки высоки: автономный транспорт, робототехника, агротех, промышленная автоматизация.

\subsection*{Избранные публикации}
\begin{itemize}
    \item \href{https://arxiv.org/abs/2501.02267}{Towards a constructive framework for control theory}
    \item \href{https://arxiv.org/abs/2105.07152}{On stochastic stabilization of sampled systems}
    \item \href{https://arxiv.org/abs/2205.13409}{On stochastic stabilization via non-smooth control Lyapunov functions}
\end{itemize}

\section{Основные научные разработки}
\subsection{CALF: безопасное обучение с подкреплением}
\textbf{Идея.} Мы соединяем алгоритмы обучения с подкреплением с «страхующим» контроллером, который отвечает за безопасность. Когда модель уверена — она берёт на себя управление, когда нет — контроль остаётся у надежного регулятора. За счёт функций Ляпунова мы формально доказываем, что система остаётся стабильной.

\textbf{Что это даёт.} В испытаниях гибридный подход демонстрировал устойчивое улучшение по сравнению с базовыми \textit{RL}-алгоритмами при сохранении совместимости с распространёнными методами (TD3, PPO и др.). Подход проверен на сложных динамических задачах (в том числе для подводных аппаратов) и переносится на промышленную робототехнику и автономный транспорт.

\subsection{Regelum: Платформа для быстрого прототипирования}
Regelum — Python фреймворк с открытым исходным кодом, предназначенный для быстрого блочного прототипирования систем управления и автоматизации. Компоненты оформлены как самодостаточные ноды с явными входами/выходами и объединяются в ориентированные графы; исполнение автоматизируется движком, который разрешает зависимости, упорядочивает вычисления и поддерживает параллелизм (на основе Dask) и иерархическую композицию вложенных графов. Фреймворк обеспечивает автоматическую обработку задержек и синхронизацию исполнения нод в графе (дискретные и непрерывные модели, гибридная динамика, ограничения реального времени) и полноценную интеграцию с ML/RL (специализированные ноды для нейросетей и оптимизаторов, адаптеры к стандартным алгоритмам). 

Перспективы развития: переход от простого DAG к декларативной узловой модели с «эффектами‑как‑данными», где узлы возвращают намерения (\textit{intents}), а планировщик детерминированно (DE, superdense time) разрешает конфликты и коммитит изменения до покоя. Это покрывает не‑DAG сценарии (например, многократные обновления в рамках одной итерации графа) через правила и приоритеты (Rete/Datalog) без ручного микропланирования.

\section{Готовые к коммерциализации проекты}
\subsection{AIDA-T: Автономный агроробот}
\textbf{Рынок:} 280 млрд руб. защищенный грунт в России, рост 12–15\% в год.

\textbf{Технические преимущества:}
\begin{itemize}
    \item Единственная автоматическая диагностика на рынке; точность 86–87\% на собственных датасетах
    \item Гибридная ходовая система
    \item Срок окупаемости: 2.5–3 года (при экономии 6–8 млн руб./год)
    \item Полное импортозамещение с высоким расширением функционала
\end{itemize}

\textbf{Финансовые показатели:}
\begin{itemize}
    \item Стоимость: 15–20 млн руб.
    \item Маржинальность: 40–45\%
    \item Прогноз выручки к 2030: 1+ млрд руб.
    \item Целевая доля рынка: 3–5\%
\end{itemize}
\textbf{Статус:} Грант Skoltech STRIP успешно завершён; прототип готов к испытаниям, уровень \textit{TRL}~5.

\section{Команда}
\subsection*{Руководитель}
Павел Осиненко --- Dr.-Ing. habil.; 25+ публикаций (Scopus Q1); 10+ лет опыта в России и Германии; экспертиза: безопасный~\textit{RL}\ и теория управления. % chktex 12

\subsection*{Научная команда}
\begin{itemize}
    \item Ключевая команда: проф. Павел Осиненко, инженер‑исследователь Илья Рякин
    \item 3 готовящихся кандидата наук (защиты 2025–2026)
    \item 5+ инженеров‑исследователей с опытом в \textit{RL}/\textit{CV}
    \item Молодые исследователи — аспиранты и магистранты Сколтеха
\end{itemize}
\textbf{Ключевые сотрудники:}
\begin{itemize}
    \item Соискатели КТН (2025–нач. 2026): Яременко Г.А., Вульф М.Д., Григулецкий М.А., Рякин И.С.
    \item Аспиранты (в т.ч. инженеры-исследователи): Давиденко С.А., Ибрагим С., Белов Д.Д., Гунявой В.
    \item Магистранты: Фук Н.С.
\end{itemize}

\section{Направления развития}
\subsection*{Готовые к развитию продукты}
\begin{itemize}
    \item \textbf{AIDA Robotics} --- агротехнологии (продукт готов к запуску; интерес от агрохолдингов)
    \item \textbf{Industrial~\textit{RL} Platform} --- \textit{B2B}, промышленная автоматизация, безопасный ИИ
    \item \textbf{Robotics Safety Suite} --- решения для робототехники (в т.ч. медицинская, сервисная)
    \item \textbf{Компьютерное зрение для автономных систем} --- детекция/сегментация/трекинг; классификация дефектов; оценка объёма и прогноз показателей (в разработке); перенос на другие домены; обучение собственных моделей (\textit{CNN}, \textit{ResNet}/\textit{EfficientNet}, \textit{U-Net}/\textit{Mask R-CNN}, \textit{YOLO}/\textit{Detectron2}); диффузия для аугментации; \textit{MLOps}/edge-деплой (\textit{Jetson})
\end{itemize}

\subsection*{Перспективные направления}
\begin{itemize}
    \item Финансовые технологии (\textit{FinTech}): применение \textit{CALF} в алготрейдинге; контролируемые риски
    \item Автономный транспорт (\textit{MobTech}): безопасные системы управления, гарантии стабильности, сертификация
    \item Большие языковые модели для автоматического управления (\textit{LLM Control}): автоматизация обучения роботов через трансляцию естественного языка в контроллеры; обеспечение безопасности; ускорение обучения навыков
    \item \textit{VR}/\textit{AI} терапевтические игры для детей с аутизмом (\textit{VR Therapy}): \textit{VR}-платформа с обучением с подкреплением для адаптивного формирования социальных и моторных навыков через геймификацию
    \item Локализационно-навигационные решения (\textit{ModuSLAM}): модульная библиотека \textit{SLAM} на фактор-графах; мультисенсорная фьюжн (камера/\textit{LiDAR}/\textit{IMU}/\textit{GPS}); оптимизация через \textit{gtsam}; поддержка \textit{ROS1/ROS2}; применение в навигации теплиц; статус --- исследовательский прототип

\end{itemize}

\section{Конкурентные преимущества}
\subsection*{Технологические}
Единственная команда в России с экспертизой в безопасном \textit{RL}; патентуемые алгоритмы и архитектурные решения; работающие прототипы; масштабируемая архитектура.

\subsection*{Рыночные}
First-mover advantage в безопасном \textit{RL}; интерес от клиентов; государственная поддержка.

\subsection*{Командные}
Уникальная экспертиза руководителя; опыт коммерциализации; партнёрства с индустрией; подготовка специалистов для автономных систем (курсы, стажировки, ротации).

\section{Финансовые перспективы}
\subsection*{Краткосрочные (1–2 года)}
\begin{itemize}
    \item AIDA-T: 50–100 млн руб. выручки
    \item Regelum: 20–30 млн руб. лицензионных доходов
    \item Консалтинг: 10–15 млн руб. проектных доходов
\end{itemize}
\subsection*{Среднесрочные (3–5 лет)}
\begin{itemize}
    \item Агротех платформа: 500 млн --- 1 млрд руб. выручки
    \item Industrial \textit{RL}: 200–300 млн руб. рекуррентных доходов
    \item IP-лицензирование: 50–100 млн руб. ежегодно
\end{itemize}
\subsection*{Долгосрочные (5+ лет)}
Готовность к стратегическим партнёрствам; экспансия в ЕС и Азию; развитие линейки продуктов.

\section{Партнерства и экосистема}
\subsection*{Научные партнёры}
Сколтех; Технический университет Хемница (Германия); международные группы в области \textit{RL}.

\subsection*{Индустриальные партнёры}
Крупные агрохолдинги; промышленные интеграторы.

\subsection*{Государственная поддержка}
Грант СТАРТ-ИИ (5 млн руб.); предыдущий грант УМНИК (500 тыс. руб.); поддержка ФСИ.

\section{Образовательная программа}
\begin{itemize}
    \item \textit{Reinforcement Learning with Safety Guarantees} --- Ляпунов, constrained \textit{RL}, \textit{CALF}; практикум в Regelum
    \item \textit{Advanced Automatic Control and Numerical Reliability} --- устойчивое управление, вычислительная надёжность, формальная верификация
    \item \textit{Planning and Decision-Making for Control and Robotics} --- планирование, поиск, безопасные контроллеры с элементами ИИ
    \item \textit{Computer Vision for Autonomous Systems} --- детекция, трекинг, навигация, \textit{SLAM}; Research Practicum: Regelum/Robotics
\end{itemize}

\section{Текущие задачи и планы}
\subsection*{Научные цели}
\begin{itemize}
    \item Развитие теории безопасного \textit{RL} с формальными гарантиями
    \item Развитие методов планирования и поддержки принятия решений
    \item Новые методы анализа динамических систем
\end{itemize}

\subsection*{Практические цели}
\begin{itemize}
    \item Завершение разработки AIDA-T для серийного производства
    \item Создание платформы Regelum для корпоративных клиентов
    \item Расширение портфеля ИС
\end{itemize}

\subsection*{Коммерческие цели}
\begin{itemize}
    \item Привлечение стратегических партнёров
    \item Развитие сети дистрибьюторов для AIDA-T
    \item Лицензионная модель для CALF-технологий
\end{itemize}

\section{Контакты}
\textbf{Павел Валерьевич Осиненко, PhD}\\
    \indent Руководитель лаборатории AIDA\\
    \indent \href{mailto:pavel.osinenko@skoltech.ru}{pavel.osinenko@skoltech.ru}

\vspace{10pt}
\noindent \textbf{Сергей Александрович Давиденко, аспирант}\\
    \indent Ключевой исполнитель проекта AIDA-T\\
    \indent \href{mailto:sergei.davidenko@skoltech.ru}{sergei.davidenko@skoltech.ru}



\end{document}
