\documentclass[12pt,a4paper]{article}
\usepackage[utf8]{inputenc}
\usepackage[russian]{babel}
\usepackage[T2A]{fontenc}
\usepackage{geometry}
\usepackage{setspace}
\usepackage{titlesec}
\usepackage{enumitem}
\usepackage{array}
\usepackage{longtable}
\usepackage{booktabs}
\usepackage[unicode,pdfencoding=auto]{hyperref}

% Настройки для корректной работы с русскими символами
\usepackage{cmap}
\DeclareUnicodeCharacter{00A0}{~}

\geometry{left=3cm,right=1.5cm,top=2cm,bottom=2cm}
\onehalfspacing

\titleformat{\section}{\large\bfseries}{\arabic{section}.}{1em}{}
\titleformat{\subsection}{\normalsize\bfseries}{\arabic{section}.\arabic{subsection}.}{1em}{}
\titleformat{\subsubsection}{\normalsize\bfseries}{\arabic{section}.\arabic{subsection}.\arabic{subsubsection}.}{1em}{}

\begin{document}

\begin{center}
\textbf{\Large ПРОЕКТ ТЕХНИЧЕСКОГО ЗАДАНИЯ} \\
\textbf{\Large на выполнение НИОКР по конкурсу «Старт-ИИ-1-2025»} \\[1em]
\textbf{\large Тема НИОКР:} \\
\textbf{\large «Разработка прототипа автономной робототехнической системы AIDA-T для интеллектуального мониторинга и диагностики томатов в промышленных теплицах»} \\[1em]
\textbf{Заявитель:} Давиденко Сергей Александрович \\
\textbf{Дата:} август 2025 г.
\end{center}

\newpage

\tableofcontents
\newpage

\section{Цель выполнения НИОКР}

Разработать и протестировать прототип автономной робототехнической системы AIDA-T (Agrobotic Intelligent Data Analyzer for Tomatoes) с интегрированными алгоритмами искусственного интеллекта для автоматизированного мониторинга и диагностики состояния томатов в промышленных теплицах, обеспечивающей точность диагностики заболеваний не менее 85\% и готовой к промышленному внедрению.

\textbf{Основные научно-технические проблемы, на решение которых направлено выполнение НИОКР:}

\begin{itemize}
\item Отсутствие мобильных робототехнических систем, способных автономно перемещаться как по бетонным покрытиям теплиц, так и по рельсовым путям между рядами растений без использования дополнительных механических приспособлений;
\item Низкая точность диагностики заболеваний растений существующими автоматизированными системами (70-80\%) по сравнению с экспертной оценкой агрономов (85-90\%);
\item Недостаточная стабильность позиционирования камерных систем на мобильных платформах (±5-10 мм на пиксель), что снижает точность анализа изображений;
\item Отсутствие комплексных решений, обеспечивающих одновременную диагностику заболеваний и оценку урожайности в едином автономном робототехническом комплексе;
\item Необходимость замещения трудозатратных рутинных методов визуального контроля агрономов с обеспечением круглосуточного мониторинга состояния растений.
\end{itemize}

\section{Назначение научно-технического продукта}

Прототип системы AIDA-T предназначен для автоматизации процессов мониторинга и диагностики состояния томатов в промышленных теплицах путем замещения рутинных методов визуального контроля, выполняемых агрономами.

\textbf{Основное функциональное назначение:}

\begin{enumerate}
\item Круглосуточная автоматизированная диагностика фитопатологических заболеваний томатов (мучнистая роса, другие грибковые и бактериальные инфекции);
\item Автоматическая оценка урожайности и контроль созревания плодов с определением объемных характеристик;
\item Непрерывный мониторинг состояния растений на всех стадиях вегетации без участия человека;
\item Раннее выявление патологий для своевременного принятия агротехнических решений.
\end{enumerate}

\textbf{Области применения прототипа:}
\begin{itemize}
\item Промышленные тепличные комплексы площадью от 1 га;
\item Агротехнические комплексы закрытого грунта;
\item Селекционные и научно-исследовательские центры;
\item Семеноводческие хозяйства.
\end{itemize}

\textbf{Целевые потребители:}
\begin{itemize}
\item Крупные агрохолдинги и тепличные комбинаты;
\item Агрономические службы предприятий;
\item Фитопатологические лаборатории;
\item Научно-исследовательские институты растениеводства.
\end{itemize}

\section{Технические требования к научно-техническому продукту}

\subsection{Основные технические параметры}

\subsubsection{Функции, выполнение которых должен обеспечивать разрабатываемый научно-технический продукт}

\textbf{1. Функция автономной навигации и перемещения:}
\begin{itemize}
\item Автоматическое движение по бетонным покрытиям теплицы с использованием меканум-колес для всенаправленного перемещения;
\item Самостоятельный подъем и движение по рельсовым путям между рядами культур с использованием специальных нейлоновых рельсовых колес;
\item Автономное переключение между режимами движения на основе анализа типа поверхности;
\item Позиционирование относительно объектов съемки с высокой точностью;
\item Преодоление препятствий и адаптация к неровностям поверхности.
\end{itemize}

\textbf{2. Функция визуального сканирования и съемки:}
\begin{itemize}
\item Регулировка высоты камерной системы в широком диапазоне для съемки растений различной высоты;
\item Стабилизированная съемка с активной компенсацией вибраций при движении по рельсам;
\item Многоракурсное сканирование растений с пересекающимися углами обзора;
\item Автоматическое слияние кадров для формирования детализированного изображения высокого разрешения.
\end{itemize}

\textbf{3. Функция интеллектуальной диагностики заболеваний:}
\begin{itemize}
\item Автоматическое распознавание симптомов мучнистой росы с использованием CNN-архитектур;
\item Выявление признаков грибковых и бактериальных инфекций на ранних стадиях;
\item Классификация степени поражения растений с определением локализации очагов заболеваний;
\item Формирование отчетов о выявленных патологиях с геопозиционной привязкой.
\end{itemize}

\textbf{4. Функция оценки урожайности:}
\begin{itemize}
\item Автоматический подсчет количества плодов на растении;
\item Определение объемных характеристик плодов с использованием стереозрения;
\item Оценка степени созревания томатов по цветовым и морфологическим признакам;
\item Прогнозирование сроков сбора урожая на основе динамики созревания.
\end{itemize}

\textbf{5. Функция сбора и обработки данных:}
\begin{itemize}
\item Создание цифровых карт состояния посадок с временными метками;
\item Формирование структурированных отчетов о выявленных проблемах;
\item Ведение базы данных мониторинга по каждому растению;
\item Экспорт данных для интеграции с информационными системами предприятия.
\end{itemize}

\textbf{6. Функция круглосуточного автономного мониторинга:}
\begin{itemize}
\item Непрерывная работа в автоматическом режиме без участия оператора;
\item Автоматическое планирование и корректировка маршрутов обхода территории;
\item Адаптация к изменяющимся условиям освещения теплицы;
\item Автономное возвращение на базовую станцию для подзарядки.
\end{itemize}

\subsubsection{Количественные параметры, определяющие выполнение научно-техническим продуктом своих функций}

\begin{enumerate}
\item \textbf{Точность диагностики мучнистой росы:} не менее 86\% при использовании CNN-модели на тестовом наборе данных объемом не менее 5000 изображений;

\item \textbf{Точность позиционирования камерной системы:} не более ±1,5 мм на пиксель при съемке в условиях движения по рельсовым путям;

\item \textbf{Погрешность оценки объема плодов:} не более 15\% при использовании параллельных алгоритмов RANSAC и PointNet на плодах томатов диаметром от 40 до 100 мм;

\item \textbf{Диапазон регулировки высоты съемки:} от 0,1 до 3,0 метров с дискретностью позиционирования не более 5 см;

\item \textbf{Время автономной работы:} не менее 8 часов непрерывной работы без подзарядки при температуре окружающей среды 18-28°C;

\item \textbf{Скорость движения:} не менее 0,5 м/с по бетонному покрытию и не менее 0,3 м/с по рельсовым путям;

\item \textbf{Объем обрабатываемого датасета:} возможность обучения CNN-модели на наборах данных объемом не менее 15000 изображений с временем обучения не более 24 часов на доступном вычислительном оборудовании.
\end{enumerate}

\subsection{Конструктивные требования к научно-техническому продукту}

\subsubsection{Внешний вид и состав научно-технического продукта}

Прототип системы AIDA-T представляет собой мобильную автономную платформу на четырех колесах с установленной телескопической камерной мачтой. Общие габариты: длина не более 1,2 м, ширина не более 0,8 м, высота в сложенном состоянии не более 0,6 м, в разложенном состоянии до 3,5 м.

\textbf{Основные функциональные части прототипа:}

\begin{enumerate}
\item \textbf{Мобильная платформа с гибридной ходовой системой:}
   \begin{itemize}
   \item Четыре меканум-колеса для всенаправленного движения по бетонному покрытию;
   \item Четыре выдвижных нейлоновых рельсовых колеса для движения по технологическим рельсам;
   \item Система активной амортизации на базе сервоприводов;
   \item Бесщеточные мотор-редукторы с энкодерами для точного позиционирования.
   \end{itemize}

\item \textbf{Телескопическая камерная мачта:}
   \begin{itemize}
   \item Вертикальная направляющая с линейным приводом для регулировки высоты;
   \item Система активной стабилизации на базе гироскопов и акселерометров;
   \item Поворотная платформа для горизонтального позиционирования камер;
   \item Защитный кожух камерного модуля с пылевлагозащитой IP63.
   \end{itemize}

\item \textbf{Система технического зрения:}
   \begin{itemize}
   \item Две промышленные Ethernet-камеры с глобальным затвором для стереозрения;
   \item Система светодиодной подсветки с автоматической регулировкой яркости;
   \item Модуль обработки изображений на базе ARM-процессора;
   \item Система калибровки и синхронизации камер.
   \end{itemize}

\item \textbf{Вычислительный модуль:}
   \begin{itemize}
   \item Одноплатный компьютер с GPU для выполнения алгоритмов машинного обучения;
   \item Модуль беспроводной связи (Wi-Fi, 4G) для удаленного мониторинга;
   \item Система хранения данных на базе SSD-накопителя;
   \item Блок управления питанием и мониторинга заряда батареи.
   \end{itemize}

\item \textbf{Система электропитания:}
   \begin{itemize}
   \item Литий-ионная аккумуляторная батарея напряжением 24В;
   \item Зарядная станция с автономной стыковкой;
   \item Система управления питанием с мониторингом энергопотребления;
   \item Блоки питания для различных подсистем (5В, 12В, 24В).
   \end{itemize}

\item \textbf{Программное обеспечение:}
   \begin{itemize}
   \item Модуль автономной навигации на базе ROS2;
   \item CNN-модели для диагностики заболеваний растений;
   \item Алгоритмы оценки объема плодов (RANSAC и PointNet);
   \item Веб-интерфейс для мониторинга и управления системой.
   \end{itemize}
\end{enumerate}

\subsubsection{Требования к конструкции и исходным компонентам}

\textbf{Требования к конструкции:}
\begin{itemize}
\item Корпус платформы должен быть изготовлен из алюминиевого профиля серии 40x40 мм с соединительными элементами из нержавеющей стали;
\item Камерная мачта должна обеспечивать продольную жесткость не менее 500 Н/мм для минимизации вибраций;
\item Все электронные компоненты должны иметь класс защиты не ниже IP63 для работы в условиях повышенной влажности теплиц;
\item Конструкция должна обеспечивать простоту обслуживания с возможностью замены основных компонентов без полной разборки системы.
\end{itemize}

\textbf{Требования к исходным компонентам:}
\begin{itemize}
\item \textbf{Меканум-колеса:} диаметр 200±5 мм, материал роликов — полиуретан твердостью 85 Shore A, нагрузка на колесо не менее 25 кг;
\item \textbf{Рельсовые колеса:} материал — нейлон PA6, диаметр 100±2 мм, профиль под рельс сечением 30x30 мм;
\item \textbf{Промышленные камеры:} разрешение не менее 5 МП, частота кадров не менее 30 fps, интерфейс Gigabit Ethernet, глобальный затвор;
\item \textbf{Вычислительная платформа:} ARM Cortex-A78 или аналогичный, GPU с поддержкой CUDA, оперативная память не менее 8 ГБ;
\item \textbf{Аккумуляторная батарея:} литий-ионная, номинальное напряжение 24В, емкость не менее 20 А·ч, количество циклов заряд-разряд не менее 1000.
\end{itemize}

\textbf{Требования к программным компонентам:}
\begin{itemize}
\item Операционная система: Ubuntu 22.04 LTS или новее с поддержкой ROS2;
\item Фреймворки машинного обучения: PyTorch 2.0+ или TensorFlow 2.10+;
\item Библиотеки компьютерного зрения: OpenCV 4.5+, PCL (Point Cloud Library);
\item При разработке не будут использованы сторонние технологии, требующие дополнительной оплаты пользователем.
\end{itemize}

\subsubsection{Требования к массогабаритным характеристикам научно-технического продукта}

\begin{itemize}
\item \textbf{Общая масса прототипа:} не более 80 кг в полной комплектации;
\item \textbf{Габаритные размеры платформы:} длина не более 1200 мм, ширина не более 800 мм, высота не более 600 мм (в транспортном положении);
\item \textbf{Максимальная высота камерной мачты:} не более 3500 мм в рабочем положении;
\item \textbf{Клиренс платформы:} не менее 50 мм для преодоления неровностей поверхности;
\item \textbf{Колесная база:} не более 700 мм для обеспечения маневренности в узких проходах теплиц.
\end{itemize}

\subsubsection{Требования к мощностным характеристикам научно-технического продукта}

\begin{itemize}
\item \textbf{Общее энергопотребление системы:} не более 300 Вт в режиме активной съемки и обработки данных;
\item \textbf{Потребляемая мощность приводов платформы:} не более 150 Вт при движении по горизонтальной поверхности;
\item \textbf{Потребляемая мощность вычислительного модуля:} не более 80 Вт при выполнении алгоритмов компьютерного зрения;
\item \textbf{Мощность системы освещения:} не более 50 Вт при максимальной яркости светодиодных модулей;
\item \textbf{Эффективность зарядной системы:} не менее 85\% при зарядке аккумуляторной батареи.
\end{itemize}

\subsubsection{Требования к удельным характеристикам научно-технического продукта}

\begin{itemize}
\item \textbf{Производительность обследования:} не менее 1000 м² тепличной площади за 8 часов работы;
\item \textbf{Удельное энергопотребление:} не более 2,4 Вт·ч на 1 м² обследованной площади;
\item \textbf{Плотность данных:} не менее 10 изображений высокого разрешения на 1 м² для обеспечения достаточной детализации анализа;
\item \textbf{Скорость обработки изображений:} не менее 5 кадров в секунду при разрешении 2592×1944 пикселей;
\item \textbf{Точность навигации:} отклонение от заданного маршрута не более 50 мм при движении по рельсовым путям.
\end{itemize}

\subsubsection{Требования к программной части аппаратно-программного комплекса}

\textbf{Требования к вычислительным ресурсам:}
\begin{itemize}
\item Процессор: ARM Cortex-A78 с частотой не менее 2,0 ГГц или x86-64 с аналогичной производительностью;
\item Оперативная память: не менее 8 ГБ DDR4 для обеспечения работы алгоритмов машинного обучения;
\item Графический процессор: GPU с поддержкой CUDA и объемом видеопамяти не менее 4 ГБ;
\item Постоянная память: SSD-накопитель объемом не менее 256 ГБ для хранения данных и программного обеспечения.
\end{itemize}

\textbf{Требования к программному обеспечению:}
\begin{itemize}
\item Модульная архитектура на базе ROS2 (Robot Operating System) для обеспечения взаимодействия подсистем;
\item Поддержка протоколов Ethernet, Wi-Fi 802.11ac, 4G LTE для передачи данных;
\item Веб-интерфейс, совместимый с браузерами Chrome, Firefox, Safari последних версий;
\item API для интеграции с внешними системами управления тепличными комплексами;
\item Система логирования событий и ошибок с возможностью удаленной диагностики.
\end{itemize}

\subsubsection{Требования к условиям апробации и тестирования научно-технического продукта}

\textbf{Условия эксплуатации прототипа:}
\begin{itemize}
\item \textbf{Температура окружающей среды:} от +10°C до +35°C (типичные условия промышленных теплиц);
\item \textbf{Относительная влажность воздуха:} до 90\% без конденсации для обеспечения работы электронных компонентов;
\item \textbf{Освещенность:} от 500 до 50000 лк с возможностью адаптации алгоритмов к изменяющимся условиям;
\item \textbf{Тип поверхности:} бетонные дорожки толщиной не менее 100 мм и металлические рельсы сечением 30x30 мм;
\item \textbf{Высота потолков теплицы:} не менее 4 метров для обеспечения работы выдвижной мачты.
\end{itemize}

\textbf{Требования к испытательной площадке:}
\begin{itemize}
\item Промышленная теплица площадью не менее 1000 м² с действующими посадками томатов;
\item Наличие комбинированной инфраструктуры: бетонные дорожки шириной не менее 1,5 м и рельсовые пути между рядами растений;
\item Возможность размещения базовой станции для подзарядки прототипа;
\item Стабильное подключение к сети Internet для удаленного мониторинга испытаний;
\item Возможность безопасного тестирования в присутствии обслуживающего персонала.
\end{itemize}

\textbf{Условия проведения испытаний:}
\begin{itemize}
\item Продолжительность испытаний: не менее 240 часов суммарного времени работы в течение 30 календарных дней;
\item Режим работы: круглосуточное тестирование с периодами активной съемки и обработки данных;
\item Контрольные измерения: верификация результатов диагностики экспертами-агрономами на выборке не менее 500 растений;
\item Сравнительный анализ: сопоставление результатов оценки урожайности с ручными измерениями на контрольной группе из 100 растений.
\end{itemize}

\subsection{Требования по патентной охране}

В рамках выполнения НИОКР планируется проведение следующих мероприятий по охране интеллектуальной собственности:

\begin{enumerate}
\item \textbf{Подача заявки на полезную модель} «Гибридная ходовая система для мобильных роботов в условиях теплиц» — срок подачи до 6 месяца выполнения НИОКР;

\item \textbf{Подача заявки на полезную модель} «Камерная мачта с активной стабилизацией для мобильных систем технического зрения» — срок подачи до 8 месяца выполнения НИОКР;

\item \textbf{Регистрация программы для ЭВМ} по алгоритмам диагностики заболеваний растений на базе CNN-архитектур — срок подачи до 10 месяца выполнения НИОКР;

\item \textbf{Рассмотрение возможности патентования} способа гибридной оценки объема плодов с использованием параллельных алгоритмов RANSAC и PointNet — решение принимается до 9 месяца выполнения НИОКР на основе результатов патентного поиска;

\item \textbf{Проведение патентного поиска} в области автономных систем мониторинга растений для подтверждения новизны технических решений — завершение до 3 месяца выполнения НИОКР.
\end{enumerate}

Все мероприятия по патентной охране будут выполняться с привлечением патентного поверенного для обеспечения высокого качества заявочных материалов и защиты интересов правообладателя.

\section{Отчетность по НИОКР}

По результатам выполнения НИОКР будет предоставлена следующая техническая документация:

\textbf{Научно-технические отчеты:}
\begin{itemize}
\item Отчет о проведенных исследованиях и разработках;
\item Отчет о результатах испытаний прототипа в полевых условиях;
\item Отчет о патентных исследованиях и охране интеллектуальной собственности.
\end{itemize}

\textbf{Эскизная конструкторская документация на прототип:}
\begin{itemize}
\item Сборочные чертежи мобильной платформы и камерной мачты;
\item Спецификации на основные узлы и компоненты системы;
\item Схемы электрические принципиальные системы управления и питания;
\item Чертежи основных оригинальных узлов (гибридная ходовая система, камерная мачта);
\item Схемы функциональные программно-аппаратного комплекса.
\end{itemize}

\textbf{Программная документация:}
\begin{itemize}
\item Алгоритмы работы программных модулей системы навигации и технического зрения;
\item Описание программного обеспечения с архитектурой системы;
\item Техническое задание на программное обеспечение;
\item Инструкция для пользователя по эксплуатации системы AIDA-T;
\item Инструкция для системного администратора по установке и настройке ПО.
\end{itemize}

\textbf{Документация по испытаниям:}
\begin{itemize}
\item Программа и методики испытаний прототипа в лабораторных условиях;
\item Программа и методики полевых испытаний в условиях промышленной теплицы;
\item Протоколы испытаний функциональных характеристик системы;
\item Протоколы испытаний точности диагностики заболеваний с экспертной верификацией;
\item Протоколы испытаний точности оценки урожайности с контрольными измерениями;
\item Сравнительный анализ результатов с существующими методами контроля.
\end{itemize}

\textbf{Дополнительная документация:}
\begin{itemize}
\item Технико-экономическое обоснование коммерциализации разработки;
\item Рекомендации по дальнейшему развитию и масштабированию системы;
\item Материалы заявок на объекты интеллектуальной собственности;
\item Акты о передаче результатов интеллектуальной деятельности.
\end{itemize}

\vfill

\begin{center}
\textbf{Разработчик проекта технического задания:} \\
Давиденко Сергей Александрович \\[1em]
\textbf{Дата:} \underline{\hspace{3cm}} \\[1em]
\textbf{Подпись:} \underline{\hspace{5cm}}
\end{center}

\end{document}