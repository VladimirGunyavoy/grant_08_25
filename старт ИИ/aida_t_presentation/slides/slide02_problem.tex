% Слайд 2: Проблема и актуальность
\slidewithimage{Проблема и актуальность}{
\textbf{\large \textcolor{aiblue}{Ключевые проблемы тепличного производства:}}
\vspace{0.5cm}

\begin{itemize}
    \item[$\times$] \textbf{Трудозатратный контроль:} Агрономы тратят до 8-12 часов на обход 1 га теплиц
    \vspace{0.3cm}
    
    \item[$\times$] \textbf{Позднее выявление болезней:} Потери урожая до 30\% при несвоевременной диагностике
    \vspace{0.3cm}
    
    \item[$\times$] \textbf{Дефицит экспертов:} Нехватка квалифицированных агрономов и фитопатологов
    \vspace{0.3cm}
    
    \item[$\times$] \textbf{Субъективность оценки:} Человеческий фактор при визуальном контроле
\end{itemize}

\vspace{0.5cm}
\textcolor{aiorange}{\textbf{Масштаб проблемы:}}
\begin{itemize}
    \item 150+ крупных тепличных комплексов в России
    \item Рынок защищенного грунта: 280 млрд ₽ (2024)
    \item Рост рынка роботизации АПК: +25-30\% в год
\end{itemize}
}{Проблемы в теплице}